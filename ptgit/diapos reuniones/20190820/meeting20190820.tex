\pdfoptionpdfminorversion=7
\documentclass{beamer}

\mode<presentation>
{
  \usetheme{Madrid}       % or try default, Darmstadt, Warsaw, ...
  \usecolortheme{default} % or try albatross, beaver, crane, ...
  \usefonttheme{serif}    % or try default, structurebold, ...
  \setbeamertemplate{navigation symbols}{}
  \setbeamertemplate{caption}[numbered]
}

\usepackage{amsmath}
\usepackage[utf8x]{inputenc}
\usepackage{listings}
\usepackage{lmodern}

\usetheme{default}

\title{Proyecto de titulo I}
\author{Yerko Zec}
\institute[]{FI - UNAB}
\date{2019/08/20}


\begin{document}

\begin{frame}[plain]
  \titlepage
\end{frame}

\addtocounter{framenumber}{-1}

\begin{frame}{Estudio bibliografico}
\begin{itemize}
 \item Se realizó una búsqueda de papers relacionados a detección de colisiones en poliedros 
 \item Se encontraron 3 papers relacionados los cuales debido al tiempo no se alcanzó a leer más que el abstract
\end{itemize}
\end{frame}


\begin{frame}{Problemas}
 \begin{itemize}
  \item Primer problema observado fue la nula organización de los tiempos para la realización de la actividad solicitada
  \item Debido a la poca organización de los tiempos al ocurrir un problema de compilación se gastó tiempo que no se tenía en la solución de ese problema.
 \end{itemize}

\end{frame}


\begin{frame}{ToDo}
\begin{itemize}
 \item Definitivamente la mejor organización de los tiempos para la realización de las actividades
 \item Configurar kile de tal manera que la compilación de los archivos no tenga problemas
\end{itemize}
\end{frame}

\end{document}
